\documentclass{article}
\usepackage[utf8]{inputenc}
\usepackage{lmodern,textcomp}
\usepackage{tabto}
\usepackage{multirow}
\usepackage{tgheros}
\usepackage[left=3cm,right=2cm,top=3cm,bottom=2cm]{geometry}
\usepackage{outline} 
\usepackage{pmgraph} 
\usepackage[normalem]{ulem}
\usepackage{tikz}
\usepackage{graphicx}
\usepackage{tikzpagenodes}
\usepackage{verbatim}
\usepackage{svg}
\pagestyle{plain}
\title{
\includegraphics[scale=0.5]{CodeSix LogoTypeCut.png} \\
\vspace*{1in}
{\Huge \textbf{Documento Candidatura}}\\
\vspace*{0.25in}
\textbf{Progetto Ingegneria Del Software}\\
\vspace{0.1in}
}

\author{
    \begin{tabular}[t]{c@{\extracolsep{8em}}c} 
        Albertin Enrico  & Marcatti Pietro \\
        Davide Spada & Marco Andrea Limongelli \\ 
        Bettin Michele & Matteo Raccanello \\
    \end{tabular}
    \vspace*{0.5in} \\
    Dipartimento di Matematica \\
    \textbf{Università degli studi di Padova} \\
    \begin{tikzpicture}[remember picture,overlay]
     \node[anchor=south] at (current page text area.south) {\includegraphics[height=1in]{logo_unipd.png}};
    \end{tikzpicture}
} 
\date{\today}


%--------------------Make usable space all of page
\setlength{\oddsidemargin}{0in} \setlength{\evensidemargin}{0in}
\setlength{\topmargin}{0in}     \setlength{\headsep}{-.25in}
\setlength{\textwidth}{6.5in}   \setlength{\textheight}{8.5in}
%--------------------Indention
\setlength{\parindent}{1cm}

\begin{document}
\maketitle
\newpage

{\fontfamily{qhv}\selectfont
\section{Resoconto incontro con il proponente}
L’incontro con il proponente è stato fissato via e-mail per il giorno \textbf{4 Novembre 2021} per una chiamata \textit{Zoom} con il Signor Gregorio Piccoli.\newline
Di seguito un breve resoconto dell’incontro. \newline
L’incontro ha avuto inizio alle \textbf{16:30}, con presenti tutti e sei i membri del gruppo e Gregorio Piccoli, proponente del capitolato per Zucchetti S.P.A., Durante la chiamata sono state poste domande da parte del gruppo al signor Piccoli riguardo aspetti generali e tecnici del progetto come:
\begin{itemize}
    \item I \textbf{casi d’uso} del prodotto
    \item Il \textbf{deploy} del prodotto finale
    \item La scelta dei \textbf{metodi di visualizzazione} dei dati, l’origine e il trattamento degli stessi
    \item Applicazioni e spunti di riflessione derivate dai \textbf{requisiti opzionali}
    \item Approfondimenti circa la \textbf{libreria di visualizzazione} richiesta dal capitolato
    \item Il \textbf{front-end} del prodotto stesso \newline
\end{itemize}
L’incontro si è concluso positivamente alle 17:05 e i membri del gruppo si sono trattenuti in chiamata fino alle 18:15 per riflettere e fissare gli spunti emersi dalla discussione con il proponente.
}
\end{document}
