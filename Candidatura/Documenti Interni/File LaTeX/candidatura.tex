\documentclass{article}
\usepackage[utf8]{inputenc}
\usepackage{lmodern,textcomp}
\usepackage{tabto}
\usepackage{multirow}
\usepackage{tgheros}
\usepackage[left=3cm,right=2cm,top=3cm,bottom=2cm]{geometry}
\usepackage{outline} 
\usepackage{pmgraph} 
\usepackage[normalem]{ulem}
\usepackage{tikz}
\usepackage{graphicx}
\usepackage{tikzpagenodes}
\usepackage{verbatim}
\usepackage{svg}
\pagestyle{plain}
\title{
\includegraphics[scale=0.5]{CodeSix LogoTypeCut.png} \\
\vspace*{1in}
{\Huge \textbf{Documento Candidatura}}\\
\vspace*{0.25in}
\textbf{Progetto Ingegneria Del Software}\\
\vspace{0.1in}
}

\author{
    \begin{tabular}[t]{c@{\extracolsep{8em}}c} 
        Albertin Enrico  & Marcatti Pietro \\
        Davide Spada & Marco Andrea Limongelli \\ 
        Bettin Michele & Matteo Raccanello \\
    \end{tabular}
    \vspace*{0.5in} \\
    Dipartimento di Matematica \\
    \textbf{Università degli studi di Padova} \\
    \begin{tikzpicture}[remember picture,overlay]
     \node[anchor=south] at (current page text area.south) {\includegraphics[height=1in]{logo_unipd.png}};
    \end{tikzpicture}
} 
\date{\today}


%--------------------Make usable space all of page
\setlength{\oddsidemargin}{0in} \setlength{\evensidemargin}{0in}
\setlength{\topmargin}{0in}     \setlength{\headsep}{-.25in}
\setlength{\textwidth}{6.5in}   \setlength{\textheight}{8.5in}
%--------------------Indention
\setlength{\parindent}{1cm}

\begin{document}
\maketitle
\newpage
{\fontfamily{qhv}\selectfont


{\Huge \textbf{Documento Candidatura}}
\section{Scelta del capitolato}
Con la presente il gruppo CodeSix intende comunicare la decisione di realizzare il progetto descritto nel capitolato d’appalto C5, denominato:\tab\textbf{Login Warrior}\newline
commissionato dall’azienda Zucchetti S.P.A.\newline

\noindent In seguito alla pubblicazione dei capitolati d’appalto, il gruppo si è riunito per discutere in merito alla fattibilità dei diversi progetti proposti. Basandosi su un carico di lavoro di 100 ore per ogni componente del gruppo e quindi per un totale di 600 ore produttive per la realizzazione di un progetto, il gruppo ha ritenuto i capitolati C1 e C2 non compatibili con le proprie disponibilità, in quanto le tempistiche per la realizzazione degli stessi risultavano troppo elevate. \newline

\noindent Per quanto riguarda i progetti descritti nei capitolati C3, C4 e C6, seppur trovandoli interessanti, il gruppo non ha inteso candidarsi per la realizzazione in quanto nessuno di questi ha ottenuto l’unanimità sulla decisione di realizzarlo, cosa che al contrario è avvenuta per il progetto descritto nel capitolato C5.\newline

\noindent La scelta di quest ultimo è individuabile principalmente nella rilevanza che sempre più la sicurezza informatica e la presenza corretta sul \textit{web} rivestono nel settore dei servizi terziari e terziari avanzati. Questo capitolato costituisce un’ottima palestra per maturare delle competenze tecniche altamente spendibili nel mondo professionale quali la capacità di creare \textit{web app} responsive e la capacità di lavorare con enormi \textit{dataset} multidimensionali. Inoltre, i membri del team hanno riconosciuto come estremamente allettante la possibilità di lavorare al fianco di una azienda così rilevante sul territorio nazionale.\newline

\noindent Il gruppo ha quindi organizzato un incontro virtuale, tramite una chiamata \textit{Zoom}, con il Signor Gregorio Piccoli - proponente del capitolato per \textit{Zucchetti S.P.A} - durante il quale ogni componente ha fatto domande di carattere generale come esempi di casi d’uso del prodotto, e domande di carattere tecnico in merito al \textit{deploy} del prodotto finale, le librerie da utilizzare e le tecnologie preferibili per la realizzazione della \textit{web application}.\newline

\noindent Concluso l'incontro, volendo garantire la consegna di un prodotto finale di qualità e volendo realizzare un progetto che fosse stimolante per tutti i componenti, il gruppo ha preso la decisione di realizzare il progetto descritto nel capitolato C5 e quindi di proporsi con la seguente candidatura.


\section{Impegni}
Viene giustificata nella sezione seguente il prezzo preventivato per la realizzazione del progetto, tenendo a mente il minimo ammesso di € 13.000 che calibrato su un gruppo di sei persone ammonta a € 11.143.\newline

\noindent Il tempo rendicontato esclude la fase di candidatura e parte dall'acquisizione dell’appalto in data \textbf{19 Novembre 2021} passando per tre fasi di revisione ( RTB, PB e CA) e terminando con l'ultima di queste in cui viene fissata la scadenza massima di consegna per il giorno \textbf{31 marzo 2022}.\newline 

\noindent Il preventivo dei costi è stato calcolato sulla base del monte ore e sulla retribuzione associata ai ruoli. \newline Ogni fase è riassunta tramite delle tabelle che esprimono le ore produttive e il costo orario e terminano con una tabella riassuntiva che le sommarizza.
Poiché il totale delle ore produttive assegnate a persona deve essere nell’intervallo 80-100, il progetto equivale a un lavoro previsto, dato il numero dei partecipanti, di circa \textbf{550-600 ore}.\newline

\noindent Di seguito sono elencati i ruoli che verranno ricoperti a rotazione da tutti i membri del team e le relative retribuzioni orarie:
\begin{itemize}
    \item Responsabile\tab 30 €/h
    \item Amministratore\tab 20 €/h
    \item Analista\tab 25 €/h
    \item Progettista\tab 25 €/h
    \item Programmatore\tab 15 €/h
    \item Verificatore\tab 15 €/h
\end{itemize}

\vspace{1em}
 \hskip-0.2cm
\def\arraystretch{1.5}%
\begin{tabular}{ |p{0.8cm}|p{2.5cm}|p{0.8cm}|p{0.8cm}|p{0.8cm}|p{0.8cm}|p{1.3cm}|p{0.8cm}|p{3cm}|}
 \hline
 \multicolumn{8}{|c|}{Ruoli} & \\
 \hline
 \multirow{7}{3em}{Nome} 
 & \textbf{Ore assegnate} & Resp. & Amm. & An. & Prg. & Program. & Ver. & \textbf{Totale Ore Persona}\\ \cline{2-9}
 & Marco Andrea Limongelli & 9 & 12 & 10 & 15 & 28 & 26 & 100\\ \cline{2-9}
 & Davide Spada & 7 & 6 & 9 & 18 & 29 & 31 & 100\\ \cline{2-9}
 & Pietro Marcatti & 10 & 8 & 9 & 13 & 35 & 25 & 100\\ \cline{2-9}
 & Michele Bettin & 8 & 7 & 8 & 15 & 30 & 32 & 100\\ \cline{2-9}
 & Enrico Albertin & 8 & 10 & 10 & 13 & 35 & 24 & 100\\ \cline{2-9}
 & Matteo Raccanello & 6 & 15 & 9 & 15 & 26 & 29 & 100\\ \cline{1-9}
 \multicolumn{2}{|c|}{\textbf{Totale Ore Ruoli}} & \textbf{48} & \textbf{58} & \textbf{55} & \textbf{89} & \textbf{183} & \textbf{167} & \textbf{600}\\ \hline
\end{tabular}
\vspace{2em}

 \noindent Applicando al monte ore appena illustrato le retribuzioni orarie di cui sopra otteniamo la giustificazione del prezzo richiesto per la realizzazione del progetto:
 
\vspace{2em}
\hskip -0.2cm
\begin{tabular}{|c|c|c|}
    \hline
     Ruolo &  Ore & Costo (€)\\\hline
     Responsabile & 48 & 1440\\\hline
     Amministratore & 58 & 1160\\\hline
     Analista & 55 & 1375\\\hline
     Progettista & 89 & 2225\\\hline
     Programmatore & 183 & 2745\\\hline
     Verificatore & 167 & 2505\\\hline
     \textbf{Totale} & \textbf{600} & \textbf{11450}\\\hline
\end{tabular}


}

\end{document}
